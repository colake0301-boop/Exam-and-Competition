\documentclass[12pt, a4paper]{article}

% --- Packages ---
\usepackage[utf8]{inputenc}
\usepackage[T1]{fontenc}
\usepackage{amsmath, amssymb, amsthm}
\usepackage{geometry}
\usepackage{fancyhdr}
\usepackage{enumitem} % Key package for spacing control

% --- Page Layout ---
\geometry{left=2.5cm, right=2.5cm, top=2.5cm, bottom=2.5cm}
\setlength{\parindent}{0pt}
\setlength{\parskip}{1em}

% --- Headers/Footers ---
\pagestyle{fancy}
\fancyhf{}
\lhead{\textbf{MAT3006 Real Analysis}}
\chead{Final Exam}
\rhead{Fall 2025}
\cfoot{Problem \theprobcount \quad -- \quad Page \thepage}

% --- Math Macros ---
\newcommand{\R}{\mathbb{R}}
\newcommand{\dx}{\, dx}
\newcommand{\mstar}{m^*} 
\newcommand{\abs}[1]{\left| #1 \right|}
\newcommand{\norm}[1]{\left\| #1 \right\|}

% =========================================================
%   1. THE "PROBLEM" ENVIRONMENT (One Page Per Problem)
% =========================================================
\newcounter{probcount}

\newenvironment{problem}{%
    \clearpage % Force new page
    \stepcounter{probcount}%
    \noindent \textbf{\Large Problem \theprobcount.}%
    \par \vspace{0.2cm}%
}{%
    \vfill % Push footer to bottom
    \clearpage
}

% =========================================================
%   2. THE "PARTS" LIST (Fixes Vertical Spacing)
% =========================================================
\newlist{parts}{enumerate}{1}
\setlist[parts]{
    label=(\roman*), 
    leftmargin=1cm, 
    labelsep=0.5em,
    topsep=0pt,     % <--- This eliminates the "strange" gap
    partopsep=0pt,
    parsep=1em      % Space between parts (i) and (ii)
}

\title{\vspace{-2cm} \textbf{Final Exam: MAT3006 Real Analysis}}
\date{December 8, 2025}

\begin{document}

% --- Cover Page ---
\maketitle
\thispagestyle{empty}

\hrule
\vspace{1em}
\textbf{Instructions:}
\begin{itemize}
    \item Attempt all problems.
    \item Each problem is presented on a separate page.
    \item Justify your answers with rigorous proofs.
    \item Unless otherwise stated, $m$ denotes the Lebesgue measure.
\end{itemize}
\vspace{1em}
\hrule
\vspace{1em}
\begin{center}
    \textit{(Turn the page to begin)}
\end{center}

% --- Problem 1 ---
\begin{problem}
    \begin{parts}
        \item Does there exist a pair of sets $E$ and $G$ such that $m^*(E) = m^*(G)$ but $m^*(G \setminus E) > 0$? Justify your answer.
        
        \item Suppose $E \subset \R$ is a set of positive outer measure ($m^*(E) > 0$). Show that for every $\alpha \in (0, 1)$, there exists an open interval $I$ such that:
        \[
        m^*(E \cap I) \ge \alpha \cdot m^*(I).
        \]
    \end{parts}
\end{problem}

% --- Problem 2 (Updated as requested) ---
\begin{problem}
    \begin{parts}
        \item \textbf{Prove or Disprove:} If $f$ and $g$ are absolutely continuous functions on $[a,b]$, then their product $f \cdot g$ is absolutely continuous on $[a,b]$.
        
        \item \textbf{Prove or Disprove:} If $f, g$ are absolutely continuous on $[a,b]$, then the integration by parts formula holds:
        \[
        \int_{a}^{b} f'(x)g(x) \dx = f(b)g(b) - f(a)g(a) - \int_{a}^{b} f(x)g'(x) \dx.
        \]
    \end{parts}
\end{problem}

% --- Problem 3 ---
\begin{problem}
    Let $f$ be a function of bounded variation on $[a,b]$. Let $TV(f)$ denote the total variation of $f$ on $[a,b]$.

    \begin{parts}
        \item Show that:
        \[
        \int_{a}^{b} |f'(x)| \dx \le TV(f).
        \]
        
        \item Show that the equality holds in the inequality above if and only if $f$ is absolutely continuous on $[a,b]$.
    \end{parts}
\end{problem}

% --- Problem 4 ---
\begin{problem}
    \begin{parts}
        \item Let $f$ be an absolutely continuous function on $[a,b]$.
        \textbf{Prove or Disprove:} Given $\epsilon > 0$, there exists $\delta > 0$ such that for \textbf{any} countable collection of disjoint open intervals $(a_k, b_k) \subset [a,b]$ satisfying $\sum (b_k - a_k) < \delta$, we have:
        \[
        \sum_{k=1}^{\infty} |f(b_k) - f(a_k)| < \epsilon.
        \]
        
        \item Show that an increasing absolutely continuous function maps sets of measure zero to sets of measure zero.
    \end{parts}
\end{problem}

% --- Problem 5 ---
\begin{problem}
    Let $f$ be a bounded measurable function on a set $E$ . 
    Show that there exist a sequence of continuous functions $\{f_n\}$ converges to $f$ pointwise a.e.
\end{problem}

% --- Problem 6 ---
\begin{problem}
    \begin{parts}
        \item Let $f \in L^1([0,1])$ with $f(x) > 0$ and $\int_{0}^{1} f(x) \dx = 1$.
        \textbf{Prove or Disprove:} 
        \[
        \int_{0}^{1} \log f(x) \dx \le 0.
        \]
        
        \item Let $f, g$ be non-negative, bounded measurable functions on $[a,b]$ with $fg \geq 1$
        \textbf{Prove or Disprove:}
        \[
        \left(\int_{0}^{1} f \right) \left(\int_{0}^{1} g \right) \ge 1.
        \]
    \end{parts}
\end{problem}

% --- Problem 7 ---
\begin{problem}
    Let $h \in L^\infty(\R)$ be a periodic function with period 1 such that $\int_{0}^{1} h(x) \dx = 0$. Define the sequence $f_n(x) = h(nx)$ for $x \in [0,1]$.

    \begin{parts}
        \item Does $\{f_n\}$ have a subsequence that converges weakly in $L^1$? If it does, what is the limit?
        \item Does $\{f_n\}$ have a subsequence that converges pointwise  a.e.? If it does, what is the limit?
        \item Does $\{f_n\}$ have a subsequence that converges in measure? If it does, what is the limit?
    \end{parts}
\end{problem}

% --- Problem 8 ---
\begin{problem}
    Let $1 < p_1 < p_2 < \infty$. Suppose $\{f_n\}$ is a sequence of functions bounded in $L^{p_2}([a,b])$ (i.e., $\sup_n \|f_n\|_{p_2} < \infty$) and $f_n \to f$ in measure.

    \vspace{1em}

    \textbf{Prove or Disprove:} There exists a subsequence $\{f_{n_k}\}$ converges to $f$ in $L^{p_1}([a,b])$.
\end{problem}

% --- Problem 9 ---
\begin{problem}
    \begin{parts}
        \item Let $\{f_n\} \subset L^{\infty}([a,b])$. If $\int_{a}^{x} f_n(t) \, dt \to 0$ for all $x \in [a,b]$, can we conclude that $\int_{a}^{b} f_n g \to 0$ for any $L^{1}([a,b])$ function $g$?
        
        \item Let $\{f_n\} \subset L^1([a,b])$. If $\int_{a}^{x} f_n(t) \, dt \to 0$ for all $x \in [a,b]$, can we conclude that $\int_{a}^{b} f_n g \to 0$ for any $g \in L^1([a,b])$?
    \end{parts}
\end{problem}

\end{document}