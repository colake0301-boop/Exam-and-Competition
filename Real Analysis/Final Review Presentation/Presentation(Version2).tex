\documentclass[english, aspectratio=169]{beamer}
\usepackage{hyperref}
\usefonttheme{serif}

% Packages
\usepackage{latexsym,amsmath,xcolor,multicol,booktabs,calligra}
\usepackage{graphicx,pstricks,listings,stackengine}
\usepackage{amssymb}
\usepackage{mathrsfs}
\usepackage{CUHKSZ}
% --- COLOR DEFINITIONS ---
\xdefinecolor{cuhksz1}{rgb}{0.457,0.0585,0.42578}      % Purple
\xdefinecolor{cuhksz2}{rgb}{0.86328,0.63671875,0}      % Gold
\xdefinecolor{cuhksz3}{rgb}{0.953125,0.87109,0.6875}   % Beige
\definecolor{deepred}{rgb}{0.6,0,0}

\lstset{
    basicstyle=\ttfamily\small,
    keywordstyle=\bfseries\color{cuhksz1},
    emphstyle=\ttfamily\color{deepred},
    stringstyle=\color{cuhksz2},
    numbers=left,
    numberstyle=\small\color{cuhksz3},
    rulesepcolor=\color{red!20!green!20!blue!20},
    frame=shadowbox,
}

% --- METADATA ---
\author{Cola Ke}
\title{Final Review: Real Analysis}
\subtitle{Borel-Cantelli, Differentiation, $L^p$ Spaces, and Modes of Convergence}
\institute{The Chinese University of Hong Kong (Shenzhen)}
\date{December 6, 2025}

% --- COMMANDS ---
\newcommand{\R}{\mathbb{R}}
\newcommand{\dx}{\,dx}
\newcommand{\norm}[1]{\left\lVert#1\right\rVert}

\begin{document}

% --- TITLE PAGE ---
\begin{frame}
    \titlepage
\end{frame}

\begin{frame}{Outline}
    \tableofcontents
\end{frame}

% ==============================================================================
% TOPIC 1
% ==============================================================================
\section{Topic 1: Borel-Cantelli \& Convergence}

\begin{frame}{Topic 1: Overview}
    \textbf{Key Concept:} The Borel-Cantelli Lemma.
    
    It is the primary tool for translating conditions on integrals or measures into pointwise almost everywhere (a.e.) convergence results.
    
    \vspace{1em}
    \textit{"The message is that 'Borel-Cantelli' way of arguing is important!"}
\end{frame}

% --- Problem 1 ---
\begin{frame}{Problem 1}
    \begin{block}{Problem}
    Let $\{f_k\}_{k=1}^\infty$ be a sequence of integrable functions on a measurable set $E$. Suppose that for every $k$:
    $$ \int_E |f_{k+1} - f_k| < \delta_k^2 $$
    where $\sum_{k=1}^\infty \delta_k < \infty$.
    
    \textbf{Prove:} The sequence $\{f_k\}$ converges pointwise almost everywhere to some function $f$.
    \end{block}
\end{frame}

\begin{frame}{Solution to Problem 1}
    \textbf{Strategy:}
    In order to show $f_k$ converges a.e., it suffices to show that $\sum |f_{k+1} - f_k| < \infty$ a.e.
    
    \begin{itemize}
        \item Let $A_k = \{ x \in E \mid |f_{k+1}(x) - f_k(x)| > \delta_k \}$.
        \item By Chebyshev's Inequality: $m(A_k) \le \frac{1}{\delta_k} \int_E |f_{k+1} - f_k| < \delta_k$.
        \item Since $\sum \delta_k < \infty$, by the Borel-Cantelli Lemma, almost every $x$ belongs to only finitely many $A_k$.
    \end{itemize}
    \textit{Note: This is the key step in proving the Riesz-Fischer Theorem.}
\end{frame}

% --- Problem 2 ---
\begin{frame}{Problem 2}
    \begin{block}{Problem}
    Let $\{f_k\}_{k=1}^\infty$ be a sequence of integrable functions on a measurable set $E$. Suppose that for every $k$:
    $$ \int_E |f_{k+1} - f_k| < \delta_k $$
    where $\sum_{k=1}^\infty \delta_k < \infty$.
    
    \textbf{Prove:} The sequence $\{f_k\}$ converges pointwise almost everywhere to some function $f$.
    \end{block}
\end{frame}

\begin{frame}{Remark on Problem 2}
    \textbf{Comparison:}
    The proof from Problem 1 (using Chebyshev directly on the sets) works for Problem 1 but \textbf{not} for Problem 2 directly because the power of convergence is different.
    
    \textbf{Alternative Argument:}
    Consider the function $g(x) = \sum_{k=1}^\infty |f_{k+1}(x) - f_k(x)|$.
    \begin{itemize}
        \item By the Monotone Convergence Theorem (MCT), $\int_E g = \sum \int_E |f_{k+1} - f_k| < \sum \delta_k < \infty$.
        \item Thus $g(x)$ is finite a.e., implying the series converges absolutely a.e.
        \item This implies $f_k \to f$ a.e. (The "Beppo-Levi" argument).
    \end{itemize}
\end{frame}

% --- Problem 3 ---
\begin{frame}{Problem 3}
    \begin{block}{Problem}
    Let $\{f_n\}_{n=1}^\infty$ be a sequence of integrable functions on $\mathbb{R}$ such that:
    $$ \int_{\mathbb{R}} |f_n(x) - f(x)| \dx \le \frac{1}{n^{1+\epsilon}} $$
    \begin{enumerate}
        \item[(a)] If $\epsilon > 0$, prove that $f_n \to f$ pointwise almost everywhere.
        \item[(b)] If $\epsilon = 0$, provide a counterexample to show that pointwise a.e. convergence may fail.
    \end{enumerate}
    \end{block}
\end{frame}

\begin{frame}{Solution to Problem 3}
    \textbf{Part (a) $\epsilon > 0$:}
    Since $\sum \frac{1}{n^{1+\epsilon}} < \infty$, we can sum the integrals. By the Beppo-Levi argument (similar to Problem 2), the series $\sum |f_n - f|$ converges a.e., which implies $|f_n - f| \to 0$ a.e.

    \textbf{Part (b) $\epsilon = 0$:}
    Consider the "Typewriter Sequence" (moving characteristic functions):
    $$ \chi_{[0, 1/2]}, \chi_{[1/2, 1]}, \chi_{[0, 1/4]}, \chi_{[1/4, 1/2]}, \dots $$
    The integral decays like $1/n$ (harmonic series diverges), and the "bump" keeps cycling through the interval infinitely often, so it diverges everywhere.
\end{frame}

% --- Problem 4 ---
\begin{frame}{Problem 4}
    \begin{block}{Problem}
    A sequence $\{f_n\}$ of measurable functions is said to be \textbf{Cauchy in Measure} if for every $\eta > 0$ and $\epsilon > 0$, there exists an integer $N$ such that for all $n, m \ge N$:
    $$ m(\{ x \in E \mid |f_n(x) - f_m(x)| \ge \eta \}) < \epsilon $$
    
    \textbf{Prove:} If $\{f_n\}$ is Cauchy in measure, then there exists a subsequence $\{f_{n_k}\}$ that converges pointwise almost everywhere.
    \end{block}
\end{frame}

\begin{frame}{Remark on Problem 4}
    This is a standard homework result. The key takeaway is utilizing the \textbf{Borel-Cantelli} style of argument to extract the subsequence efficiently.
\end{frame}

% --- Problem 5 ---
\begin{frame}{Problem 5 (Diophantine Approximation)}
    \begin{block}{Problem}
    \textbf{(a) Order 2 is Universal:} 
    Use the Pigeonhole Principle to prove Dirichlet's Approximation Theorem: For any irrational $x$, there exist infinitely many rational numbers $p/q$ such that $|x - p/q| < 1/q^2$.

    \textbf{(b) Order $>2$ is Rare:}
    Prove that the set of $x \in [0,1]$ such that $|x - p/q| < 1/q^3$ for infinitely many rationals $p/q$ has Lebesgue measure zero.
    \end{block}
\end{frame}

\begin{frame}{Solution to Problem 5(b)}
    \textbf{Proof:}
    It suffices to show by the Borel-Cantelli Lemma that the sum of the measures of the target intervals is finite.
    
    $$ \sum_{q=1}^\infty m\left( \bigcup_{p=0}^q \left\{x \in [0,1] : \left|x - \frac{p}{q}\right| < \frac{1}{q^3} \right\} \right) $$
    
    The measure at each level $q$ is roughly $q \cdot \frac{2}{q^3} = \frac{2}{q^2}$. Since $\sum \frac{1}{q^2} < \infty$, the set of such $x$ has measure 0.
\end{frame}

% ==============================================================================
% TOPIC 2
% ==============================================================================
\section{Topic 2: Differentiation Theory}

\begin{frame}{Topic 2: Overview}
    \textbf{Key Questions:}
    \begin{enumerate}
        \item Which classes of functions are differentiable almost everywhere? 
        \item Under what conditions does the Fundamental Theorem of Calculus (FTC) hold? i.e., $\int_a^b f' = f(b) - f(a), \quad \frac{d}{dx} \int_{a}^{x} f= f $.
    \end{enumerate}

    \textbf{Summary:}
    \begin{itemize}
        \item \textbf{Monotone Functions:} Differentiable a.e. (Proof uses Vitali Covering Lemma).
        \item \textbf{Bounded Variation (BV):} Differentiable a.e., but FTC may fail (e.g., Cantor-Lebesgue function).
        \item \textbf{Absolute Continuity (AC):} The necessary and sufficient condition for FTC.
    \end{itemize}
\end{frame}

\begin{frame}{Properties of Absolutely Continuous Functions}
    \begin{itemize}
        \item \textbf{Equivalence with Stronger Definition:} \\
        The standard definition of absolute continuity (requiring $\sum |f(b_k) - f(a_k)| < \epsilon$) is equivalent to the stronger condition:
        $$ \sum_{k=1}^n TV(f|_{[a_k, b_k]}) < \epsilon $$
        whenever $\sum (b_k - a_k) < \delta$.

        \item \textbf{Indefinite Integral Characterization:} \\
        A function $f$ is absolutely continuous on $[a,b]$ if and only if it is an indefinite integral:
        $$ f(x) = f(a) + \int_a^x g(t) \, dt $$
        for some integrable function $g$ (specifically, $g = f'$ a.e.).

    \end{itemize}
\end{frame}

\begin{frame}{Properties of Absolutely Continuous Functions}
    \begin{itemize}
        \item \textbf{FTC for Increasing Functions:} \\
        An increasing continuous function $f$ is absolutely continuous if and only if it satisfies the Fundamental Theorem of Calculus:
        $$ \int_a^b f'(x) \, dx = f(b) - f(a) $$
    \end{itemize}
\end{frame}

% --- Problem 6 ---
\begin{frame}{Problem 6}
    \begin{block}{Problem}
    Let $\alpha, \beta > 0$. Define $f: [0,1] \to \mathbb{R}$ by:
    $$ f(x) = \begin{cases} x^\alpha \sin(\frac{1}{x^\beta}) & 0 < x \le 1 \\ 0 & x = 0 \end{cases} $$
    
    \begin{enumerate}
        \item[(i)] Prove that if $\alpha > \beta$, then $f$ is of Bounded Variation (BV) on $[0,1]$.
        \item[(ii)] Prove that if $\alpha \le \beta$, then $f$ is \textbf{not} of Bounded Variation on $[0,1]$.
    \end{enumerate}
    \end{block}
\end{frame}

\begin{frame}{Solution to Problem 6}
    \textbf{(i) Case $\alpha > \beta$:}
    $f$ is continuously differentiable on $(0,1]$. We calculate the derivative:
    $$ |f'(x)| \le \alpha x^{\alpha-1} + \beta x^{\alpha-\beta-1} $$
    Since $\alpha > \beta$, the power $\alpha-\beta-1 > -1$. Thus $|f'|$ is integrable near 0.
    Therefore, $TV(f) = \int_0^1 |f'| < \infty$.

    \textbf{(ii) Case $\alpha \le \beta$:}
    We construct a partition where the variation diverges.
    Choose points $x_n$ where $\sin(1/x^\beta) = \pm 1$. The sum of oscillations behaves like the harmonic series $\sum \frac{1}{n}$ (or worse), which diverges to infinity.
\end{frame}

% --- Problem 7 ---
\begin{frame}{Problem 7}
    \begin{block}{Problem}
    Let $\text{Lip}$, $AC$, and $BV$ denote the spaces of Lipschitz, Absolutely Continuous, and Bounded Variation functions on $[a,b]$, respectively.
    
    Prove the strict inclusions:
    $$ \text{Lip} \subsetneq AC \subsetneq BV $$
    \end{block}
\end{frame}

\begin{frame}{Solution to Problem 7}
    \begin{itemize}
        \item $\text{Lip} \subsetneq AC$: Consider $f(x) = \sqrt{x}$ on $[0,1]$. It is AC (integral of $1/2\sqrt{x}$) but not Lipschitz (derivative is unbounded at 0).
        \item $AC \subsetneq BV$: Consider the Cantor-Lebesgue function. It is BV (since it is monotone) but not AC (it maps a set of measure 0 to a set of measure 1).
    \end{itemize}
\end{frame}

% --- Problem 8 ---
\begin{frame}{Problem 8}
    \begin{block}{Problem}
    Prove that if $f$ is an absolutely continuous function, then $f$ maps measurable sets to measurable sets.
    \end{block}
\end{frame}

\begin{frame}{Solution to Problem 8}
    \textbf{Step 1:} Show $f$ satisfies the condition ($N$): it maps sets of measure 0 to sets of measure 0.
    For any $\epsilon$, cover the zero-measure set $Z$ with intervals of small total length. By absolute continuity, the image of these intervals has small total length.

    \textbf{Step 2:} General Measurable Set $E$.
    Write $E = F \cup Z$ where $F$ is an $F_\sigma$ set (countable union of closed sets) and $m(Z)=0$.
    \begin{itemize}
        \item $f(F)$ is measurable (continuous image of an $F_\sigma$ set is $F_\sigma$, hence Borel).
        \item $f(Z)$ has measure 0 by Step 1.
    \end{itemize}
    Thus $f(E) = f(F) \cup f(Z)$ is measurable.
\end{frame}

% --- Problem 9 ---
\begin{frame}{Problem 9}
    \begin{block}{Problem}
    Let $f$ be a function of bounded variation on $[a,b]$, and define the total variation function $V(x) = TV(f|_{[a,x]})$.
    
    \begin{enumerate}
        \item[(i)] Show that $|f'(x)| \le V'(x)$ almost everywhere, and $\int_a^b |f'(x)| \dx \le TV(f)$.
        \item[(ii)] Prove that $V'(x) = |f'(x)|$ a.e. if and only if $f$ is absolutely continuous.
    \end{enumerate}
    \end{block}
\end{frame}

% --- Problem 10 ---
\begin{frame}{Problem 10}
    \begin{block}{Problem}
    Let $f: [a,b] \to \mathbb{R}$ be an increasing function.
    
    \textbf{Prove:} $f$ is singular (i.e., $f'=0$ a.e.) if and only if:
    For every $\epsilon > 0$, there exists a finite disjoint collection of intervals $\{(a_k, b_k)\}$ such that:
    1. $\sum (b_k - a_k) < \epsilon$ (Small total length in domain)
    2. $\sum (f(b_k) - f(a_k)) > f(b) - f(a) - \epsilon$ (Captures most of the growth in range).
    \end{block}
\end{frame}

\begin{frame}{Remark on Problem 10}
    \textbf{Intuition:}
    If $f$ is AC, it increases "gradually" everywhere.
    If $f$ is singular and increasing, it must accomplish a "big increase" within a "very small region" (effectively on a set of measure zero).
\end{frame}

% ==============================================================================
% TOPIC 3
% ==============================================================================
\section{Topic 3: Approximation \& Separability}

\begin{frame}{Topic 3: Overview}
    \textbf{Density:}
    \begin{itemize}
        \item Simple functions, Step functions, and Continuous functions with compact support ($C_c$) are dense in $L^p(\R)$ for $1 \le p < \infty$.
    \end{itemize}

    \textbf{Separability:}
    \begin{itemize}
        \item $L^p$ is separable for $1 \le p < \infty$.
        \item $L^\infty$ is \textbf{not} separable.
        \item $C[a,b]$ is separable (by the Stone-Weierstrass Theorem).
    \end{itemize}
\end{frame}

% --- Problem 11 ---
\begin{frame}{Problem 11}
    \begin{block}{Problem (Weak Compactness)}
    Let $1 < p < \infty$. Prove that if $\{f_n\}$ is a bounded sequence in $L^p(E)$, then $\{f_n\}$ possesses a subsequence that converges in the \textbf{weak sense}.
    \end{block}
\end{frame}

% --- Problem 12 ---
\begin{frame}{Problem 12}
    \begin{block}{Problem}
    Let $1 < p < \infty$. Prove that a bounded sequence $\{f_n\}$ in $L^p(E)$ converges weakly to $f$ if and only if for every measurable subset $A \subseteq E$:
    $$ \lim_{n \to \infty} \int_A f_n = \int_A f $$
    \end{block}
\end{frame}

% --- Problem 13 ---
\begin{frame}{Problem 13}
    \begin{block}{Problem (Riemann-Lebesgue Lemma)}
    Let $f \in L^1([a,b])$. Prove that:
    $$ \lim_{n \to \infty} \int_a^b f(x) \sin(nx) \dx = 0 $$
    \end{block}
\end{frame}

% --- Problem 14 ---
\begin{frame}{Problem 14}
    \begin{block}{Problem}
    Provide an explicit example of a sequence that converges weakly to 0 but does not converge strongly (in norm) to 0.
    \end{block}
\end{frame}

\begin{frame}{Solution to Problem 14}
    \textbf{Example:} The Rademacher sequence $R_n(x) = \text{sgn}(\sin(2^n \pi x))$ on $[0,1]$.
    Or simply $f_n(x) = \sin(nx)$.
    
    \begin{itemize}
        \item $f_n \rightharpoonup 0$ (weakly) by the Riemann-Lebesgue Lemma.
        \item However, $\norm{f_n}_p$ does not go to 0 (it stays constant for Rademacher, or constant average for sin), so it does not converge strongly.
    \end{itemize}
\end{frame}

% ==============================================================================
% TOPIC 4
% ==============================================================================
\section{Topic 4: Modes of Convergence}

\begin{frame}{Topic 4: Overview}
    We study the relationships between:
    \begin{enumerate}
        \item Convergence in $L^p$ (Norm)
        \item Convergence in Measure
        \item Pointwise Convergence a.e.
        \item Weak Convergence
    \end{enumerate}
\end{frame}

\begin{frame}
    \begin{center}
        % \resizebox{width}{height}{content} - '!' maintains aspect ratio
        \resizebox{\textwidth}{!}{%
            \renewcommand{\arraystretch}{1.2} % Increase vertical spacing slightly for readability
            \setlength{\tabcolsep}{3pt}       % Reduce horizontal padding between columns
            \begin{tabular}{|l|c|c|c|c|}
                \hline
                \multicolumn{5}{|c|}{\textbf{Does Property A imply a Subsequence has Property B?}} \\
                \multicolumn{5}{|c|}{\small (Assume $m(E) < \infty$ and sequence is bounded in $L^p$)} \\
                \hline
                % Shortened headers to save width
                \textbf{Prop A $\downarrow$ \textbackslash \ Prop B $\rightarrow$} & \textbf{Strong ($L^p$)} & \textbf{Weak ($L^p$)} & \textbf{Ptwise a.e.} & \textbf{Measure} \\
                \hline
                \multicolumn{5}{|c|}{\textit{\textbf{Case 1: Reflexive Range ($1 < p < \infty$)}}} \\
                \hline
                \textbf{Strong} & YES & YES & YES & YES \\
                \hline
                \textbf{Weak} & \textcolor{red}{NO} & YES & \textcolor{red}{NO} & \textcolor{red}{NO} \\
                \hline
                \textbf{Pointwise a.e.} & \textcolor{red}{NO} & \textbf{YES} & YES & YES \\
                \hline
                \textbf{Measure} & \textcolor{red}{NO} & \textbf{YES} & YES & YES \\
                \hline
                \multicolumn{5}{|c|}{\textit{\textbf{Case 2: Non-Reflexive ($p = 1$)}}} \\
                \hline
                \textbf{Pointwise a.e.} & \textcolor{red}{NO} & \textcolor{red}{\textbf{NO}} & YES & YES \\
                \hline
                \textbf{Measure} & \textcolor{red}{NO} & \textcolor{red}{\textbf{NO}} & YES & YES \\
                \hline
            \end{tabular}%
        }
    \end{center}
\end{frame}

% --- Problem 15 ---
\begin{frame}{Problem 15}
    \begin{block}{Problem}
    Let $\{f_n\}$ be a bounded sequence in $L^p(E)$ such that $f_n \to f$ pointwise almost everywhere.
    
    \begin{enumerate}
        \item[(i)] If $p > 1$, prove that $f_n \rightharpoonup f$ (converges weakly).
        \item[(ii)] If $p = 1$, prove or disprove that $f_n \rightharpoonup f$.
    \end{enumerate}
    \end{block}
\end{frame}

\begin{frame}{Solution to Problem 15}
    \textbf{(i) Case $p > 1$: YES.}
    By weak compactness (reflexivity of $L^p$), any subsequence contains a further weakly convergent subsequence. Since the pointwise limit is $f$, the weak limit must be unique and equal to $f$. Thus the entire sequence converges weakly to $f$.

    \textbf{(ii) Case $p = 1$: NO.}
    Counter-example: $f_n = n \chi_{[0, 1/n]}$.
    \begin{itemize}
        \item Bounded in $L^1$ ($\norm{f_n}_1 = 1$).
        \item Converges pointwise a.e. to 0.
        \item Test against $g \equiv 1 \in L^\infty$: $\int f_n g = 1 \not\to 0$. Thus it does not converge weakly to 0.
    \end{itemize}
\end{frame}

\begin{frame}{Problem 16}
    \begin{block}{Problem}
    Assume $E$ has finite measure and $1 \leq p < \infty$. Suppose $\{f_n\}$ is a sequence of measurable functions that converges pointwise a.e. on $E$ to $f$.
    
    \vspace{0.5em}
    Show that $\{f_n\} \to f$ in $L^p(E)$ if there is a $\theta > 0$ such that $\{f_n\}$ belongs to and is bounded as a subset of $L^{p+\theta}(E)$.
    \end{block}
\end{frame}

\begin{frame}{Solution to Problem 16}
    \begin{proof}[Proof Idea]
    \begin{itemize}
        \item \textbf{Step 1 (Boundedness):} Since $\{f_n\}$ is bounded in $L^{p+\theta}$, Fatou's Lemma implies $f \in L^{p+\theta}$. Thus, $\|f_n - f\|_{p+\theta} \leq K$ for some constant $K$.
        
        \item \textbf{Step 2 (Holder's Inequality):} For any subset $B \subset E$, we use Hölder's inequality with exponents $r = \frac{p+\theta}{p}$ and $r'$:
        \[
        \int_B |f_n - f|^p \leq \left( \int_B |f_n - f|^{p+\theta} \right)^{\frac{p}{p+\theta}} |B|^{\frac{\theta}{p+\theta}} \leq K^p |B|^{\frac{\theta}{p+\theta}}
        \]
        Since $\theta > 0$, the integral can be made arbitrarily small by choosing $|B|$ small (Uniform Integrability).
        
        \item \textbf{Step 3 (Egorov's Theorem):} Given $\epsilon > 0$, split $E$ into a set $A$ where $f_n \to f$ uniformly, and a small set $B$ (where $|B|$ controls the integral above).
    \end{itemize}
    \end{proof}
\end{frame}

% --- Problem 16 ---
\begin{frame}{Problem 17}
    \begin{block}{Problem (Radon-Riesz Property)}
    Let $1 \le p < \infty$. Suppose $\{f_n\} \subset L^p(E)$ converges pointwise almost everywhere to $f \in L^p(E)$.
    
    \textbf{Prove:} $f_n \to f$ in $L^p$ (Strongly) if and only if $\lim_{n \to \infty} \norm{f_n}_p = \norm{f}_p$.
    \end{block}
\end{frame}

\begin{frame}{Remark on Problem 17}
    The "$\Rightarrow$" direction is trivial by continuity of the norm.
    The "$\Leftarrow$" direction is non-trivial. It usually involves applying Fatou's Lemma to the sequence:
    $$ g_n = 2^{p-1}(|f_n|^p + |f|^p) - |f_n - f|^p $$
    Since $g_n \ge 0$, applying Fatou yields the result.
\end{frame}

% ==============================================================================
% TOPIC 5
% ==============================================================================
\section{Topic 5: Counter-Example Library}

\begin{frame}{Topic 5: Overview (Standard Library)}
    It is hard to remember every counter-example, but useful to keep a "Standard Library" to test hypotheses:
    
    \begin{enumerate}
        \item \textbf{Cantor-Lebesgue Function:} (Continuous, Monotone, derivative 0 a.e., not AC).
        \item \textbf{Typewriter Sequence:} (Converges in measure, but diverges pointwise everywhere).
        \item \textbf{The Spikes:} $n\chi_{[0, 1/n]}$ (Converges pointwise to 0, but Integral stays 1).
        \item \textbf{Rademacher Functions:} (Weak convergence to 0, no strong convergence).
    \end{enumerate}
\end{frame}

% --- Problem 17 ---
\begin{frame}{Problem 18}
    \begin{block}{Problem}
    Construct a sequence of Riemann integrable functions $f_n: [a,b] \to \mathbb{R}$ such that $\{f_n\}$ converges pointwise to a function $f$, but $f$ is \textbf{not} Riemann Integrable.
    \end{block}
\end{frame}

\begin{frame}{Remark on Problem 18}
    \textbf{Example:} Enumerate rationals in $[0,1]$ as $\{r_1, r_2, \dots\}$. Let $f_n$ be the characteristic function of $\{r_1, \dots, r_n\}$.
    Then $f_n \to \chi_{\mathbb{Q}}$, which is not Riemann integrable (Dirichlet function).
    
    \textit{Note:} If the sequence is uniformly bounded, the limit is always \textbf{Lebesgue Integrable} by the Dominated Convergence Theorem.
\end{frame}

% --- Problem 18 ---
\begin{frame}{Problem 19}
    \begin{block}{Problem}
    Construct an absolutely continuous, strictly increasing function $f: [0,1] \to \mathbb{R}$ for which $f'(x) = 0$ on a set of positive measure.
    \end{block}
\end{frame}

\begin{frame}{Solution to Problem 19}
    This requires a "Fat Cantor Set" construction (a Cantor-like set with positive measure). 
    We construct a homeomorphism that maps the Fat Cantor set (measure $>0$) to a regular Cantor set (measure 0) or vice versa, ensuring strict monotonicity while keeping the derivative zero on the "gaps".
\end{frame}

\end{document}