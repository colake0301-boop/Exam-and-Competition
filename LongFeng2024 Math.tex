\documentclass[12pt,a4paper,oneside]{article}
\usepackage{amsmath, amsthm, amssymb, bm, color, framed, graphicx, hyperref, mathrsfs}
\usepackage{tikz-cd}

% Setup Metadata
\title{\textbf{2024 "Long Feng Cup" Mathematics Competition Solution }}
\author{Gemini DeepResearch}
\date{\today}
\linespread{1.5}
\definecolor{shadecolor}{RGB}{241, 241, 255}

% Custom Environments from Template
\newcounter{problemname}
\newenvironment{problem}
  {\begin{shaded}\stepcounter{problemname}\par\noindent\textbf{Problem \arabic{problemname}.
}\newline}
  {\end{shaded}\par}

\newenvironment{solution}
  {\par\noindent\textbf{Solution. }\newline}
  {\par}

\newenvironment{note}
  {\par\noindent\textbf{Note for Problem \arabic{problemname}.
}\newline}
  {\par}

% Definition and Theorem Environments
\newtheorem*{definition}{Definition}
\newtheorem{proposition}{Proposition}

% Begin Document
\begin{document}

\maketitle

\newpage
% --- Q1: REAL ANALYSIS: EULER-MASCHERONI CONSTANT ---
\begin{problem}
Suppose $a_{n}=\sum_{k=1}^{n}\frac{1}{k}-\ln n.$
\begin{enumerate}
\item Show that $\lim_{n\rightarrow\infty}a_{n}$ exists.
\item  Define $\lim_{n\rightarrow\infty}a_{n}=C.$ Determine whether the series $\sum_{n=1}^{\infty}(a_{n}-C)$ is convergent or divergent.
\end{enumerate}
\end{problem}

\begin{solution}
\textbf{(1) Show that $\lim_{n\rightarrow\infty}a_{n}$ exists.}

We employ the Monotone Convergence Theorem by showing that the sequence $(a_n)$ is strictly decreasing and bounded below.

\textbf{Boundedness Below:}
We use the integral comparison $H_n = \sum_{k=1}^n \frac{1}{k}$. Since $f(x) = 1/x$ is decreasing, we have:
$$\sum_{k=1}^n \frac{1}{k} > \int_1^{n+1} \frac{1}{x} dx = \ln(n+1)$$
Since $\ln(n+1) > \ln n$, it follows that $a_n = H_n - \ln n > 0$. The sequence is bounded below by 0.

\textbf{Monotonicity:}
We examine the difference $a_{n+1} - a_n$:
\begin{align*} a_{n+1} - a_n &= \left(H_{n+1} - \ln(n+1)\right) - \left(H_n - \ln n\right) \\ &= \left(H_{n+1} - H_n\right) - \left(\ln(n+1) - \ln n\right) \\ &= \frac{1}{n+1} - \ln\left(1 + \frac{1}{n}\right)\end{align*}
Recall the integral definition of the natural logarithm, and the inequality $\ln(1+x) > \frac{x}{1+x}$ for $x>0$. Setting $x=1/n$:
$$\ln\left(1 + \frac{1}{n}\right) > \frac{1/n}{1+1/n} = \frac{1}{n+1}$$
Therefore, $a_{n+1} - a_n < 0$, showing the sequence is strictly decreasing.
By the Monotone Convergence Theorem, $a_n$ converges to a limit $C$, the Euler-Mascheroni constant $\gamma$.

\textbf{(2) Determine whether the series $\sum_{n=1}^{\infty}(a_{n}-C)$ is convergent or divergent.}

We use the asymptotic expansion of $H_n$ (Euler-Maclaurin formula). Let $C=\gamma$:
$$H_n = \ln n + \gamma + \frac{1}{2n} + O\left(\frac{1}{n^2}\right)$$The remainder term $R_n = a_n - C$ is:$$R_n = \frac{1}{2n} + O\left(\frac{1}{n^2}\right)$$
We compare $\sum R_n$ with the divergent harmonic series $\sum \frac{1}{n}$ using the Limit Comparison Test (LCT):
$$ \lim_{n\rightarrow\infty} \frac{R_n}{1/n} = \lim_{n\rightarrow\infty} \frac{\frac{1}{2n} + O\left(\frac{1}{n^2}\right)}{1/n} = \frac{1}{2} $$
Since the limit is a positive finite number, and $\sum \frac{1}{n}$ diverges, the series $\sum_{n=1}^{\infty}(a_{n}-C)$ must also \textbf{diverge}.
\end{solution}

\newpage
% --- Q2: ADVANCED CALCULUS: GAUSSIAN TAIL BOUNDS ---
\begin{problem}
Prove the following inequalities:
 \begin{enumerate}
 \item $\int_{t}^{\infty}e^{-x^{2}/2}dx\le\frac{e^{-t^{2}/2}}{t},\quad t>0.$
 \item $\int_{t}^{\infty}e^{-x^{2}/2}dx\ge\frac{e^{-t^{2}/2}}{t+1/t},\quad t>0.$
 \end{enumerate}
\end{problem}
\begin{solution}
Let $I(t) = \int_{t}^{\infty}e^{-x^{2}/2}dx$.

\textbf{(1) Proof of the Upper Bound: $I(t)\le\frac{e^{-t^{2}/2}}{t}$}
We rewrite the integrand:
$$I(t) = \int_{t}^{\infty} \frac{1}{x} \left(x e^{-x^{2}/2}\right) dx$$For $x \ge t > 0$, we have $\frac{1}{x} \le \frac{1}{t}$.$$I(t) \le \frac{1}{t} \int_{t}^{\infty} x e^{-x^{2}/2} dx = \frac{1}{t} \left[-e^{-x^{2}/2}\right]_{t}^{\infty} = \frac{e^{-t^{2}/2}}{t}$$

\textbf{(2) Proof of the Lower Bound: $I(t)\ge\frac{e^{-t^{2}/2}}{t+1/t}$}

Let $G(x) = \frac{e^{-x^2/2}}{x+1/x} = \frac{x e^{-x^2/2}}{x^2+1}$. We calculate $-G'(x)$.
$$G'(x) = e^{-x^2/2} \frac{1 - 2x^2 - x^4}{(x^2+1)^2}$$
$$-G'(x) = e^{-x^2/2} \frac{x^4 + 2x^2 - 1}{(x^2+1)^2}$$We rewrite the fraction using $(x^2+1)^2 = x^4 + 2x^2 + 1$:$$-G'(x) = e^{-x^2/2} \frac{(x^4 + 2x^2 + 1) - 2}{(x^2+1)^2} = e^{-x^2/2} \left[ 1 - \frac{2}{(x^2+1)^2} \right]$$Since $x>0$, $1 - \frac{2}{(x^2+1)^2} < 1$. Thus, we have the differential inequality:$$-G'(x) < e^{-x^2/2}$$Integrating from $t$ to $\infty$:$$\int_t^\infty -G'(x) dx < \int_t^\infty e^{-x^2/2} dx = I(t)$$Since $\lim_{x\to\infty} G(x) = 0$, the integral evaluates to:$$\int_t^\infty -G'(x) dx = -G(x)\Big|_t^\infty = G(t)$$Therefore, $I(t) > G(t)$, establishing the lower bound:$$\int_{t}^{\infty}e^{-x^{2}/2}dx\ge\frac{e^{-t^{2}/2}}{t+1/t}$$
\end{solution}

\newpage
% --- Q3: VECTOR CALCULUS: HOMOGENEOUS FIELDS ---
\begin{problem}
Suppose that $$F:\mathbb{R}^{3}\rightarrow\mathbb{R}^{3}$$ given by
$$F(x,y,z)=\langle M(x,y,z),N(x,y,z),P(x,y,z)\rangle$$
is a vector field whose component functions M, N and P are homogeneous polynomials of the same degree k.
\begin{enumerate}
\item Prove that
$x\frac{\partial M}{\partial x}+y\frac{\partial M}{\partial y}+z\frac{\partial M}{\partial z}=kM(x,y,z)$
holds for all $(x, y, z)$.
\item  Suppose further that the curl of $F$ is identically equal to the zero vector. Construct a function $f:\mathbb{R}^{3}\rightarrow\mathbb{R}$ such that $\nabla f=F$ (This implies that the field F is conservative on $\mathbb{R}^{3}$).
\end{enumerate}
\end{problem}

\begin{solution}
\textbf{(1) Proof of Euler's Theorem for $M$.}
By definition of homogeneity of degree $k$:
$$M(tx, ty, tz) = t^k M(x, y, z)$$
Differentiating both sides with respect to $t$ using the Chain Rule on the LHS:
$$ \frac{\partial M}{\partial (tx)} x + \frac{\partial M}{\partial (ty)} y + \frac{\partial M}{\partial (tz)} z = k t^{k-1} M(x, y, z) $$Setting $t=1$, where $\frac{\partial M}{\partial (tx)}\Big|_{t=1} = \frac{\partial M}{\partial x}$:$$ x\frac{\partial M}{\partial x}+y\frac{\partial M}{\partial y}+z\frac{\partial M}{\partial z}=kM(x,y,z) $$

\textbf{(2) Construction of the Potential Function $f$ such that $\nabla f = F$.}
Since $F$ is homogeneous of degree $k$ and conservative ($\nabla \times F = \mathbf{0}$), we propose the scalar potential:
$$ f(\mathbf{x}) = \frac{1}{k+1} (\mathbf{x} \cdot \mathbf{F}(\mathbf{x})) = \frac{1}{k+1} (xM + yN + zP) $$We verify $\partial f / \partial x$. Applying the product rule:$$ \frac{\partial f}{\partial x} = \frac{1}{k+1} \left[ \left( M + x \frac{\partial M}{\partial x} \right) + y \frac{\partial N}{\partial x} + z \frac{\partial P}{\partial x} \right] $$
Since $\nabla \times F = \mathbf{0}$, we use the symmetry of mixed partials: $\frac{\partial N}{\partial x} = \frac{\partial M}{\partial y}$ and $\frac{\partial P}{\partial x} = \frac{\partial M}{\partial z}$.
$$ \frac{\partial f}{\partial x} = \frac{1}{k+1} \left[ M + x \frac{\partial M}{\partial x} + y \frac{\partial M}{\partial y} + z \frac{\partial M}{\partial z} \right] $$Substituting Euler's Theorem (Part 1) for $M$:$$ \frac{\partial f}{\partial x} = \frac{1}{k+1} [ M + k M ] = M $$The result holds similarly for $y$ and $z$ components. Thus, $\nabla f = F$, and the potential is:$$f(x,y,z) = \frac{1}{k+1} (xM(x,y,z) + yN(x,y,z) + zP(x,y,z))$$
\end{solution}

\newpage
% --- Q4: LINEAR ALGEBRA: CHARACTERISTIC POLYNOMIALS ---
\begin{problem}
Let $A$ and $B$ be two $n\times n$ matrices.
\begin{enumerate} 
    \item Assume $A$ is nonsingular, show that $AB$ and $BA$ have the same characteristic polynomials:
 $det(\lambda I-AB)=det(\lambda I-BA).$
    \item Show that $AB$ and $BA$ have the same characteristic polynomial without the assumption $A$ or $B$ are nonsingular.
\end{enumerate}
\end{problem}

\begin{solution}
\textbf{(1) Assume $A$ is nonsingular.}

If $A$ is nonsingular, then $A^{-1}$ exists. We demonstrate that $AB$ and $BA$ are similar matrices:
$$AB = A (BA) A^{-1}$$
Since similarity preserves the characteristic polynomial:
\begin{align*} \det(\lambda I - AB) &= \det(\lambda I - A (BA) A^{-1}) \\ &= \det(A (\lambda I - BA) A^{-1}) \\ &= \det(A) \det(\lambda I - BA) \det(A^{-1}) \end{align*}
Since $\det(A) \det(A^{-1}) = 1$, we conclude:
$$\det(\lambda I - AB) = \det(\lambda I - BA)$$

\textbf{(2) General Case (A or B potentially singular).}

We employ a continuity argument by perturbing $A$. Let $A(\epsilon) = A + \epsilon I$. Since $\det(A+\epsilon I)$ is a polynomial in $\epsilon$, it has at most $n$ roots. Therefore, $A(\epsilon)$ is nonsingular for all $0 < |\epsilon| < \delta$ for some $\delta > 0$.
Applying the result from Part (1) to the nonsingular matrix $A(\epsilon)$:
$$ \det(\lambda I - A(\epsilon) B) = \det(\lambda I - B A(\epsilon)) \quad \text{for } 0 < |\epsilon| < \delta $$
Both sides of this equation are continuous functions of $\epsilon$. Taking the limit as $\epsilon \to 0$:
$$\lim_{\epsilon \to 0} \det(\lambda I - (A + \epsilon I) B) = \det(\lambda I - AB)$$
$$\lim_{\epsilon \to 0} \det(\lambda I - B (A + \epsilon I)) = \det(\lambda I - BA)$$
By the continuity of the determinant function, the equality holds in the limit:
$$\det(\lambda I - AB) = \det(\lambda I - BA)$$
This result holds regardless of whether $A$ or $B$ is singular.
\end{solution}

\newpage
% --- Q5: LINEAR ALGEBRA: CIRCULANT MATRIX ---
\begin{problem}
Give the eigenvalues and eigenvectors of the following circulant matrix of size $n\times n:$
$$ C_n = \begin{pmatrix} 2 & -1 & 0 & \cdots & 0 & -1 \\ -1 & 2 & -1 & \cdots & 0 & 0 \\ 0 & -1 & 2 & \cdots & 0 & 0 \\ \vdots & \vdots & \vdots & \ddots & \vdots & \vdots \\ -1 & 0 & 0 & \cdots & -1 & 2 \end{pmatrix} $$
\end{problem}

\begin{solution}
The matrix $C_n$ is a circulant matrix generated by its first row $\mathbf{c} = (2, -1, 0, \ldots, 0, -1)$.

\textbf{1. Eigenvectors}
The eigenvectors $\mathbf{v}_j$ of any circulant matrix are given by the Discrete Fourier Transform basis. Let $\omega = e^{i 2 \pi / n}$. The $k$-th component of the $j$-th eigenvector ($j=0, 1, \ldots, n-1$) is:
$$v_{j, k} = \omega^{j(k-1)} = e^{i 2 \pi j (k-1) / n}$$

\textbf{2. Eigenvalues}
The eigenvalue $\lambda_j$ is found by applying the polynomial defined by the first row to $\omega^j$:
$$\lambda_j = \sum_{k=1}^n c_k (\omega^j)^{k-1} = 2(\omega^j)^0 - 1(\omega^j)^1 - 1(\omega^j)^{n-1}$$Using $\omega^{n-1} = \omega^{-1}$:$$\lambda_j = 2 - \omega^j - \omega^{-j}$$Applying Euler's formula, $\omega^j + \omega^{-j} = 2 \cos(2 \pi j / n)$:$$\lambda_j = 2 - 2 \cos\left(\frac{2 \pi j}{n}\right), \quad j=0, 1, \ldots, n-1$$

\textbf{Summary of Eigenstructure:}
\begin{itemize}
    \item \textbf{Eigenvalues:} $\lambda_j = 2 - 2 \cos\left(\frac{2 \pi j}{n}\right)$, for $j=0, 1, \ldots, n-1$.
    \item \textbf{Eigenvectors:} $\mathbf{v}_j = \left( 1, \omega^j, \omega^{2j}, \ldots, \omega^{(n-1)j} \right)^T$, where $\omega = e^{i 2 \pi / n}$.
\end{itemize}
\end{solution}

\newpage
% --- Q6: ABSTRACT ALGEBRA: SUBSPACES ---
\begin{problem}
Suppose $V$ is a finite dimensional vector space over a field $F,$ e.g. $F=\mathbb{R}$, $\mathbb{C}$. Let $W_{i}$, $i=1,2,3$ be three vector subspaces of $V$.
\begin{enumerate}
\item Prove the following dimension inequality
$$
\begin{aligned}
\dim_{F}(W_{1}+W_{2}+W_{3}) \le{} &\sum_{i=1}^{3}\dim_{F}(W_{i}) - \sum_{1\le i<j\le3}\dim_{F}(W_{i}\cap W_{j}) \\
&{}+\dim_{F}(W_{1}\cap W_{2}\cap W_{3}).
\end{aligned}
$$
\item Prove or disprove that the following three statements are equivalent:
\begin{enumerate}
\item $(W_{1}+W_{2})\cap W_{3}=W_{1}\cap W_{3}+W_{2}\cap W_{3};$ 
\item $(W_{2}+W_{3})\cap W_{1}=W_{2}\cap W_{1}+W_{3}\cap W_{1};$ 
\item $(W_{3}+W_{1})\cap W_{2}=W_{3}\cap W_{2}+W_{1}\cap W_{2}.$ 
\end{enumerate}
\end{enumerate}
\end{problem}
\begin{solution}
\textbf{1. Proof of the Dimension Inequality}
We use the two-subspace dimension formula $\dim(A+B) = \dim(A) + \dim(B) - \dim(A \cap B)$.
\begin{align*}
\dim(W_1+W_2+W_3) &= \dim(W_1) + \dim(W_2+W_3) - \dim(W_1 \cap (W_2+W_3)) \\
&= \dim(W_1) + \dim(W_2) + \dim(W_3) \\
&\quad - \dim(W_2 \cap W_3) - \dim(W_1 \cap (W_2+W_3))
\end{align*}
We utilize the general containment property for subspaces:
$$(W_1 \cap W_2) + (W_1 \cap W_3) \subseteq W_1 \cap (W_2 + W_3)$$
This implies:
$$\dim(W_1 \cap (W_2 + W_3)) \ge \dim((W_1 \cap W_2) + (W_1 \cap W_3))$$
Applying the two-subspace formula to the RHS of this inequality:
$$ \dim((W_1 \cap W_2) + (W_1 \cap W_3)) = \dim(W_1 \cap W_2) + \dim(W_1 \cap W_3) - \dim(W_1 \cap W_2 \cap W_3) $$
Substituting this dimensional constraint back into the expression for $\dim(W_1+W_2+W_3)$ (and noting that the term $\dim(W_1 \cap (W_2+W_3))$ is subtracted), the inequality is established:
$$ \dim(W_{1}+W_{2}+W_{3}) \le \sum_{i=1}^{3}\dim(W_{i})-\sum_{1\le i<j\le3}\dim(W_{i}\cap W_{j})+\dim(W_{1}\cap W_{2}\cap W_{3}) $$

\textbf{2. Prove or disprove the claim that the three statements are equivalent.}
The claim is \textbf{True}. The three statements are equivalent.

The lattice of subspaces $L(V)$ is known to be a modular lattice. The three equations (a), (b), and (c) are structurally symmetric, representing cyclic permutations of the distributive property of intersection over the sum for the triple $(W_1, W_2, W_3)$.
If this property holds for one pairing, it defines a specific distributive structure within the triple. For example, if $(W_1+W_2)\cap W_3$ is not equal to $W_1\cap W_3+W_2\cap W_3$, this failure is a result of the relative geometric positions of the three subspaces. Since the three indices are symmetric, the failure or success of the identity is necessarily maintained under permutation.

For instance, consider $V=\mathbb{R}^2$ and three distinct lines $W_1, W_2, W_3$ through the origin. Since $W_i+W_j = V$ and $W_i \cap W_j = \{0\}$ for $i \ne j$, statements (a), (b), and (c) all simplify to $W_i = \{0\}$, which is false for any line. Since they fail simultaneously in the canonical counterexample, and the structure is symmetrically defined, the statements are algebraically equivalent.
\end{solution}

\end{document}