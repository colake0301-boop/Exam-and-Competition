\documentclass[12pt, a4paper, oneside]{article}
\usepackage{amsmath, amsthm, amssymb, bm, color, framed, graphicx, hyperref, mathrsfs}
\usepackage{tikz-cd}

\title{\textbf{2024 "Long Feng Cup" Mathematics Competition Solution}}
\author{ChatGPT DeepResearch}
\date{\today}
\linespread{1.5}
\definecolor{shadecolor}{RGB}{241, 241, 255}
\newcounter{problemname}

% Problem environment
\newenvironment{problem}
  {\begin{shaded}\stepcounter{problemname}\par\noindent\textbf{Problem \arabic{problemname}. }\newline}
  {\end{shaded}\par}

% Solution environment
\newenvironment{solution}
  {\par\noindent\textbf{Solution. }\newline}
  {\par}


% Note environment
\newenvironment{note}
  {\par\noindent\textbf{Note for Problem \arabic{problemname}. }\newline}
  {\par}

%Definition environment
\newtheorem*{definition}{Definition}
\newtheorem{proposition}{Proposition}

\begin{document}
\maketitle
\newpage
\begin{problem}
(1) Consider the curve $C$ on the $xy$-plane given by the parametric equations 
\[
x = 3t^2,\quad y = 2t^3,\quad t \ge 0.
\]
Express the curve as a polar curve $r = f(\theta)$. 

(2) Let 
\[
f(x) = 
\begin{cases}
x^2, & x\in [0,1],\\
2-x, & x\in [1,2].
\end{cases}
\]
Compute $\displaystyle \int_{0}^{2} f(x)\,dx$.

(3) For the power series 
\[
\sum_{n=1}^{\infty} \frac{(-1)^n}{n} x^n,
\]
determine its interval of convergence.

(4) Calculate the limit 
\[
\lim_{n\to\infty} \frac{n\bigl(n^{1/n}-1\bigr)}{\ln n}.
\]

(5) Calculate the limit 
\[
\lim_{x\to 0} \frac{\frac{1}{\ln x} - x}{x-1}.
\]
\end{problem}
\begin{solution}
(1) From the parametric equations, we have 
\[
\tan\theta = \frac{y}{x} = \frac{2t^3}{3t^2} = \frac{2}{3}t,
\]
so \(t = \tfrac{3}{2}\tan\theta\) (since \(t\ge 0\) implies \(\theta\in [0,\pi/2)\)). Next, 
\[
r^2 = x^2 + y^2 = (3t^2)^2 + (2t^3)^2 = 9t^4 + 4t^6 = t^4(9 + 4t^2).
\]
Hence 
\[
r = t^2\sqrt{9 + 4t^2} = \Bigl(\frac{3}{2}\tan\theta\Bigr)^2\sqrt{9 + 4\Bigl(\frac{3}{2}\tan\theta\Bigr)^2}.
\]
We simplify inside the square root: \(4(\frac{3}{2}\tan\theta)^2 = 9\tan^2\theta\), so \(9 + 9\tan^2\theta = 9(1+\tan^2\theta)=9\sec^2\theta\). Thus 
\[
r = \frac{9}{4}\tan^2\theta \cdot \sqrt{9\sec^2\theta} 
= \frac{9}{4}\tan^2\theta \cdot 3|\sec\theta| = \frac{27}{4}\tan^2\theta\sec\theta.
\]
Since \(\theta\in[0,\pi/2)\), \(\sec\theta>0\). Therefore the polar equation is 
\[
r = \frac{27}{4} \tan^2\theta \sec\theta.
\]

(2) We split the integral according to the definition of \(f(x)\):
\[
\int_{0}^{2} f(x)\,dx = \int_{0}^{1} x^2\,dx + \int_{1}^{2} (2-x)\,dx.
\]
Compute each part:
\[
\int_{0}^{1} x^2\,dx = \left[\frac{x^3}{3}\right]_{0}^{1} = \frac{1}{3},
\quad
\int_{1}^{2} (2-x)\,dx = \left[2x - \frac{x^2}{2}\right]_{1}^{2} = (4 - 2) - (2 - \tfrac{1}{2}) = 2 - \frac{3}{2} = \frac{1}{2}.
\]
Adding them gives \(\frac{1}{3} + \frac{1}{2} = \frac{5}{6}\). 

(3) The power series is \(\sum_{n=1}^{\infty}(-1)^n x^n / n\). The radius of convergence \(R\) is found by the ratio test: 
\[
\lim_{n\to\infty} \Bigl|\frac{(-1)^{n+1}x^{n+1}/(n+1)}{(-1)^n x^n/n}\Bigr| 
= \lim_{n\to\infty} |x|\cdot\frac{n}{n+1} = |x|.
\]
Thus \(R = 1\), so the series converges for \(|x|<1\). We must check the endpoints \(x=1\) and \(x=-1\). 

- At \(x=1\), the series becomes \(\sum_{n=1}^\infty (-1)^n/n\), which converges (it is the alternating harmonic series, summing to \(-\ln 2\)). 
- At \(x=-1\), it becomes \(\sum_{n=1}^\infty (-1)^n (-1)^n /n = \sum_{n=1}^\infty 1/n\), which diverges (harmonic series). 

Therefore the interval of convergence is \((-1,1]\). 

(4) We consider \(n^{1/n} = e^{(\ln n)/n}\). For large \(n\), 
\[
n^{1/n} = 1 + \frac{\ln n}{n} + o\Bigl(\frac{\ln n}{n}\Bigr).
\]
Hence 
\[
n(n^{1/n}-1) \approx n\cdot \frac{\ln n}{n} = \ln n.
\]
Thus 
\[
\lim_{n\to\infty} \frac{n(n^{1/n}-1)}{\ln n} = 1.
\]
A more formal justification uses L'Hôpital's rule or series expansion of the exponential as above. 

(5) Rewrite \(x = e^{-t}\) with \(t\to +\infty\). Then as \(x\to 0^+\), \(\ln x = -t\), and 
\[
\frac{1}{\ln x} - x = -\frac{1}{t} - e^{-t}.
\]
Also \(x-1 = e^{-t} - 1 \to -1\). So 
\[
\lim_{x\to 0} \frac{\frac{1}{\ln x} - x}{x-1} 
= \lim_{t\to\infty} \frac{-\frac{1}{t} - e^{-t}}{e^{-t}-1}.
\]
As \(t\to\infty\), the numerator \(\to 0\) and the denominator \(\to -1\). Therefore the limit is \(0\). 
\end{solution}
\newpage
\begin{problem}
Let \(S_0\) be the largest sphere in space passing through the point \(P_0(-5,-1,6)\) such that every point \((x,y,z)\) inside \(S_0\) satisfies 
\[
x^2 + y^2 + z^2 < 136 + 2(x + 2y + 3z).
\]
Find an equation of \(S_0\).
\end{problem}
\begin{solution}
First rewrite the given inequality in a more standard form. We have
\[
x^2 + y^2 + z^2 < 136 + 2(x + 2y + 3z)
\quad\Longleftrightarrow\quad
x^2 - 2x + y^2 - 4y + z^2 - 6z < 136.
\]
Complete the squares for each variable:
\[
(x^2 - 2x +1) + (y^2 - 4y +4) + (z^2 - 6z +9) < 136 + (1+4+9).
\]
That is
\[
(x-1)^2 + (y-2)^2 + (z-3)^2 < 136 + 14 = 150.
\]
Thus the region described is the interior of the sphere with center \(O=(1,2,3)\) and radius \(\sqrt{150}\). 

We seek the largest sphere \(S_0\) that passes through \(P_0 = (-5,-1,6)\) and is contained in (or tangent to) this sphere. If \(S_0\) has center \(C\) and radius \(r\), then \(C\) must lie inside or on the larger sphere, and \(C\) must satisfy \(\|C-P_0\| = r\). For \(S_0\) to be maximal while lying inside the sphere centered at \(O\) of radius \(\sqrt{150}\), \(S_0\) will be tangent internally to the larger sphere. Hence the distance from \(C\) to \(O\) plus \(r\) equals \(\sqrt{150}\). That is:
\[
\|C - O\| + r = \sqrt{150}, 
\qquad \|C - P_0\| = r.
\]
Geometrically, the point \(C\) must lie on the line through \(O\) and \(P_0\). We can solve as follows: Let \(d = \|P_0 - O\|\). Here \(P_0-O = (-6,-3,3)\), so 
\[
d = \sqrt{(-6)^2 + (-3)^2 + 3^2} = \sqrt{54} = 3\sqrt{6}.
\]
Parameterize \(C\) on the line from \(O\) in the direction of \(P_0\): \(C = O + t(P_0 - O)\) for some \(t\). Then 
\[
\|C - O\| = |t|\,d, 
\quad \|C - P_0\| = |1-t|\,d.
\]
Since \(C\) lies between \(O\) and \(P_0\), \(t\) will be between 0 and 1. The two conditions become
\[
|t|\,d + |1-t|\,d = d \,= \sqrt{150},
\]
and \(|1-t|\,d = r\). Actually, for \(S_0\) tangent to the big sphere, we need \(\|C-O\| + r = \sqrt{150}\). Using \(\|C-O\|=|t|d\) and \(r=|1-t|d\), this equation is
\[
|t|d + |1-t|d = \sqrt{150}.
\]
Since \(d = \sqrt{54}\), we solve \(|t|+|1-t| = \sqrt{150}/\sqrt{54} = \sqrt{\tfrac{150}{54}} = \sqrt{\tfrac{25}{9}} = \frac{5}{3}.\) For \(0 \le t \le 1\), this is \(t + (1-t) = 1\), which is less than \(5/3\). Thus \(C\) must lie on the extension beyond \(O\) away from \(P_0\), meaning \(t<0\). Set \(t = -\frac{1}{3}\) (since by direct calculation one finds \(t = -1/3\) to satisfy the equation). Then
\[
C = O + \Bigl(-\frac{1}{3}\Bigr)(-6,-3,3) = (1,2,3) + \left(2,1,-1\right) = (3,3,2).
\]
We check \(r = \|C-P_0\|\). Then \(C-P_0 = (3-(-5),3-(-1),2-6) = (8,4,-4)\), so 
\[
r = \sqrt{8^2 + 4^2 + (-4)^2} = \sqrt{96} = 4\sqrt{6}.
\]
Also \(\|C-O\| = \|(3,3,2)-(1,2,3)\| = \sqrt{2^2+1^2+(-1)^2} = \sqrt{6}\). Indeed \(\sqrt{6} + 4\sqrt{6} = 5\sqrt{6} = \sqrt{150}\), so \(S_0\) is tangent as required.

Thus the equation of \(S_0\) (the sphere with center \(C=(3,3,2)\) and radius \(\sqrt{96}\)) is
\[
(x-3)^2 + (y-3)^2 + (z-2)^2 = 96.
\]
\end{solution}
\newpage
\begin{problem}
Assume \(f(x)\) is integrable on \(\mathbb{R}\). 

(1) Prove that for any real \(a\), 
\[
\int_{0}^{a} f(x)\,dx = \int_{0}^{a} f(a - x)\,dx.
\]

(2) Compute the integral 
\[
\int_{0}^{\pi} \frac{x\sin x}{1+\cos^2 x}\,dx.
\]
\end{problem}
\begin{solution}
(1) This is a standard symmetry argument. Let \(I = \int_{0}^{a} f(a-x)\,dx\). Perform the substitution \(u = a - x\). Then \(du = -dx\), and when \(x=0\), \(u=a\); when \(x=a\), \(u=0\). Thus
\[
I = \int_{x=0}^{x=a} f(a-x)\,dx = \int_{u=a}^{u=0} f(u)(-du) = \int_{u=0}^{u=a} f(u)\,du = \int_{0}^{a} f(x)\,dx.
\]
This proves the desired equality of integrals.

(2) Let 
\[
I = \int_{0}^{\pi} \frac{x\sin x}{1+\cos^2 x}\,dx.
\]
We use the substitution \(x' = \pi - x\). Observe that \(\sin(\pi - x) = \sin x\) and \(\cos(\pi - x) = -\cos x\), so \(1+\cos^2(\pi-x) = 1+\cos^2 x\). Also when \(x=0\), \(x'=\pi\); when \(x=\pi\), \(x'=0\). Hence
\[
I = \int_{0}^{\pi} \frac{(\pi - x')\sin x'}{1+\cos^2 x'}\,dx' = \int_{0}^{\pi} \frac{(\pi - x)\sin x}{1+\cos^2 x}\,dx,
\]
where in the last equality we renamed \(x'\) back to \(x\). Now add the two expressions for \(I\):
\[
2I = \int_{0}^{\pi} \frac{(\,x + (\pi - x)\,)\sin x}{1+\cos^2 x}\,dx = \int_{0}^{\pi} \frac{\pi \sin x}{1+\cos^2 x}\,dx = \pi \int_{0}^{\pi} \frac{\sin x}{1+\cos^2 x}\,dx.
\]
To evaluate the remaining integral, use the substitution \(u = \cos x\), \(du = -\sin x\,dx\). As \(x\) runs from \(0\) to \(\pi\), \(u\) runs from \(1\) to \(-1\). Thus
\[
\int_{0}^{\pi} \frac{\sin x}{1+\cos^2 x}\,dx = \int_{u=1}^{u=-1} \frac{-du}{1+u^2} 
= \int_{-1}^{1} \frac{du}{1+u^2} = \left[\arctan(u)\right]_{-1}^{1} = \arctan(1) - \arctan(-1) = \frac{\pi}{4} - \Bigl(-\frac{\pi}{4}\Bigr) = \frac{\pi}{2}.
\]
Hence 
\[
2I = \pi \cdot \frac{\pi}{2} = \frac{\pi^2}{2}, 
\]
and so 
\[
I = \frac{\pi^2}{4}.
\]
\end{solution}
\newpage
\begin{problem}
Suppose that \(F: \mathbb{R}^3 \to \mathbb{R}^3\) is given by
\[
F(x,y,z) = \langle M(x,y,z),\, N(x,y,z),\, P(x,y,z)\rangle,
\]
where \(M,N,P\) are homogeneous polynomials of the same degree \(k\). 

(1) Prove that for each component, Euler's homogeneous function relation holds. For example, show that
\[
x\frac{\partial M}{\partial x} + y\frac{\partial M}{\partial y} + z\frac{\partial M}{\partial z} = k\,M(x,y,z).
\]

(2) Suppose furthermore that \(\nabla \times F = \mathbf{0}\) (the curl of \(F\) is the zero vector). Show that \(F\) is conservative by explicitly constructing a function \(f: \mathbb{R}^3\to\mathbb{R}\) such that \(\nabla f = F\).
\end{problem}
\begin{solution}
(1) If \(M(x,y,z)\) is a homogeneous polynomial of degree \(k\), it means that \(M(\lambda x,\lambda y,\lambda z)=\lambda^k M(x,y,z)\) for all \(\lambda\). Differentiating both sides with respect to \(\lambda\) and then setting \(\lambda=1\) gives
\[
\frac{d}{d\lambda}\Bigl(M(\lambda x,\lambda y,\lambda z)\Bigr)\Big|_{\lambda=1}
= \frac{d}{d\lambda}\bigl(\lambda^k M(x,y,z)\bigr)\Big|_{\lambda=1}.
\]
The left side, by the chain rule, is 
\[
x\frac{\partial M}{\partial x}(\lambda x,\lambda y,\lambda z) + y\frac{\partial M}{\partial y}(\lambda x,\lambda y,\lambda z) + z\frac{\partial M}{\partial z}(\lambda x,\lambda y,\lambda z)\Big|_{\lambda=1},
\]
which becomes \(xM_x + yM_y + zM_z\) at \(\lambda=1\). The right side is \(k\lambda^{k-1}M(x,y,z)\big|_{\lambda=1} = kM(x,y,z)\). Equating them yields the desired identity
\[
xM_x + yM_y + zM_z = k M(x,y,z).
\]
A similar argument applies to \(N\) and \(P\) since they are also homogeneous of degree \(k\).

(2) If \(\nabla \times F = \mathbf{0}\), the field \(F\) is locally (and in this case globally) conservative. We seek a scalar potential \(f\) with \(\nabla f = \langle M,N,P\rangle\). A convenient approach is to use Euler's identity. Consider the function 
\[
f(x,y,z) = \frac{xM(x,y,z) + yN(x,y,z) + zP(x,y,z)}{k+1}.
\]
We will show that \(\partial f/\partial x = M\), and similarly for \(y\) and \(z\). Compute
\[
\frac{\partial f}{\partial x} = \frac{1}{k+1}\Bigl[ M + xM_x + yN_x + zP_x \Bigr],
\]
where subscripts denote partial derivatives. Because \(F\) is curl-free, we have \(N_x = M_y\), \(P_x = M_z\), etc. Hence
\[
xM_x + yN_x + zP_x = xM_x + yM_y + zM_z = k\,M,
\]
by the result of part (1) applied to \(M\). Therefore
\[
\frac{\partial f}{\partial x} = \frac{1}{k+1}\bigl[M + kM\bigr] = M.
\]
By symmetry (or repeating the argument cyclically), one finds \(\partial f/\partial y = N\) and \(\partial f/\partial z = P\). Thus \(\nabla f = F\), as required. This construction shows \(F\) is conservative. 
\end{solution}
\newpage
\begin{problem}
Solve the following differential equations:

(1) \(\displaystyle \frac{dy}{dx} = \frac{x^k - n y}{x},\) where \(k,n\in \mathbb{Z}\).

(2) \(\displaystyle (8x^2y - 4xy^2 - 2y^3)\,dx \;-\; (4x^3 - 4x^2y - x y^2)\,dy = 0,\) with initial condition \(y(1)=2\).
\end{problem}
\begin{solution}
(1) The equation can be written as 
\[
\frac{dy}{dx} + \frac{n}{x}\,y = x^{k-1}.
\]
This is a first-order linear ODE. The integrating factor is \(\mu(x) = x^n\). Multiply through by \(x^n\):
\[
x^n \frac{dy}{dx} + nx^{n-1}y = x^{n+k-1}.
\]
The left side is \(\frac{d}{dx}(x^n y)\). Thus
\[
\frac{d}{dx}(x^n y) = x^{n+k-1}.
\]
Integrate both sides:
\[
x^n y = \int x^{n+k-1}\,dx = \frac{x^{n+k}}{n+k} + C,
\]
for \(n+k\neq 0\). Hence
\[
y = \frac{x^k}{n+k} + Cx^{-n}.
\]
If \(n+k=0\), say \(k=-n\), one integrates \(\frac{d}{dx}(x^n y)= x^{-1}\) to get \(x^n y = \ln|x| + C\), so \(y = x^{-n}\ln|x| + Cx^{-n}\). 

(2) Rewrite the differential form:
\[
(8x^2y - 4xy^2 - 2y^3)\,dx - (4x^3 - 4x^2y - x y^2)\,dy = 0.
\]
This is a homogeneous equation (all terms are of degree 3). Use the substitution \(y = vx\). Then \(dy = v\,dx + x\,dv\). Substitute into the equation. Alternatively, one can check it is an exact differential after division or find an integrating factor. Here we try \(y = vx\):
First express \(\frac{dy}{dx} = v + x\frac{dv}{dx}\). The differential equation becomes:
\[
8x^2(vx) - 4x(vx)^2 - 2(vx)^3 - \bigl(4x^3 - 4x^2(vx) - x(vx)^2\bigr)\frac{dy}{dx} = 0.
\]
Instead of doing that, a simpler approach is to treat it as a homogeneous first-order ODE:
\[
dy/dx = \frac{8x^2y - 4xy^2 - 2y^3}{4x^3 - 4x^2y - xy^2}.
\]
Substitute \(y=vx\), then \(dy/dx = v + x dv/dx\). The right-hand side becomes
\[
\frac{8x^2(vx) - 4x(vx)^2 - 2(vx)^3}{4x^3 - 4x^2(vx) - x(vx)^2}
= \frac{2v(4x^3 - 2x^2v - xv^2)}{x(4x^2 - 4xv - v^2x)}.
\]
Simplify by canceling a factor of \(x\):
\[
\frac{dy}{dx} = \frac{2v(4 - 2v - v^2)}{4 - 4v - v^2}.
\]
Thus
\[
v + x\frac{dv}{dx} = \frac{2v(4 - 2v - v^2)}{4 - 4v - v^2}.
\]
Rearrange:
\[
x\frac{dv}{dx} = \frac{2v(4 - 2v - v^2)}{4 - 4v - v^2} - v.
\]
Combine terms over a common denominator \((4 - 4v - v^2)\):
\[
x\frac{dv}{dx} = \frac{2v(4 - 2v - v^2) - v(4 - 4v - v^2)}{4 - 4v - v^2} 
= \frac{8v -4v^2 -2v^3 - 4v +4v^2 + v^3}{4 - 4v - v^2}
= \frac{4v - v^3}{4 - 4v - v^2}.
\]
Separate variables:
\[
\frac{4 - 4v - v^2}{4v - v^3}\,dv = \frac{dx}{x}.
\]
Simplify the left side by factoring \(v\):
\[
\frac{4 - 4v - v^2}{v(4 - v^2)} = \frac{4 - 4v - v^2}{4v - v^3} 
= \frac{(4 - v^2) - 4v}{v(4 - v^2)} 
= \frac{4}{v(4-v^2)} - \frac{4v}{v(4-v^2)} - \frac{v^2}{v(4-v^2)}
= \frac{4 - 4v - v^2}{v(4 - v^2)}.
\]
Actually, it is easier to decompose directly:
\[
\frac{4 - 4v - v^2}{4v - v^3} = \frac{4 - 4v - v^2}{v(4 - v^2)}.
\]
Perform partial fraction decomposition:
\[
\frac{4 - 4v - v^2}{v(4 - v^2)} = \frac{A}{v} + \frac{Bv + C}{4 - v^2}.
\]
We solve \(4 - 4v - v^2 = A(4 - v^2) + (Bv + C)v\). Setting \(v=0\) gives \(4 = 4A\), so \(A=1\). Expand:
\[
4 - 4v - v^2 = 4A - A v^2 + Bv^2 + C v.
\]
Plug \(A=1\):
\[
4 - 4v - v^2 = 4 - v^2 + Bv^2 + Cv.
\]
Equate coefficients:
- For \(v^2\): \(-1 = -1 + B\) implies \(B=0\).
- For \(v\): \(-4 = C\).
- Constant: \(4=4\), checks out.
Thus
\[
\frac{4 - 4v - v^2}{v(4 - v^2)} = \frac{1}{v} - \frac{4}{4 - v^2} = \frac{1}{v} - \frac{1}{2-v} - \frac{1}{2+v}
\]
(using partial fractions on \(\frac{4}{4-v^2} = \frac{1}{2-v} + \frac{1}{2+v}\)). Therefore, 
\[
\int \frac{4 - 4v - v^2}{4v - v^3}\,dv = \int \left(\frac{1}{v} - \frac{1}{2-v} - \frac{1}{2+v}\right)dv.
\]
Integrate term by term:
\[
\int \frac{1}{v}\,dv = \ln|v|,\quad
\int \frac{1}{2-v}\,dv = -\ln|2-v|,\quad
\int \frac{1}{2+v}\,dv = \ln|2+v|.
\]
So the left integral is
\[
\ln|v| + \ln|2-v| - \ln|2+v| + C = \ln\Bigl|\frac{v(2-v)}{2+v}\Bigr| + C.
\]
Hence
\[
\ln\Bigl|\frac{v(2-v)}{2+v}\Bigr| = \ln|x| + C'.
\]
Exponentiating, 
\[
\frac{v(2-v)}{2+v} = Cx,
\]
for some constant \(C\). Recall \(v = y/x\). Substituting back:
\[
\frac{(y/x)\bigl(2 - y/x\bigr)}{2 + y/x} = Cx,
\]
multiply both numerator and denominator by \(x\):
\[
\frac{y(2x - y)}{x(2x + y)} = Cx.
\]
So
\[
y(2x - y) = Cx^2 (2x + y).
\]
This is the general implicit solution. Use the initial condition \(y(1)=2\): plug \(x=1,y=2\):
\[
2(2\cdot 1 - 2) = C\cdot 1^2 (2\cdot 1 + 2) \;\;\Longrightarrow\;\; 2(0) = C\cdot 4 \;\;\Longrightarrow\;\; 0 = 4C \;\;\Longrightarrow\;\; C=0.
\]
Thus the equation reduces to \(y(2x - y) = 0\). For all \(x\), this implies either \(y=0\) or \(y=2x\). The solution satisfying \(y(1)=2\) is \(y=2x\). 
\end{solution}
\newpage
\begin{problem}
Suppose \(f>0\) is continuous on \(\mathbb{R}\). Show that if 
\[
\int_{-\infty}^{+\infty} e^{-|t-x|} f(x)\,dx \le 1
\]
for every real \(t\), then for all \(a<b\),
\[
\int_{a}^{b} f(x)\,dx \;\le\; \frac{b-a+2}{2}.
\]
\end{problem}
\begin{solution}
We use the given integral inequality at two specific values of \(t\). 

First, set \(t=a\). Then for any \(x\),
\[
e^{-|a-x|} = 
\begin{cases}
e^{-(x-a)}, & x \ge a,\\
e^{-(a-x)}, & x < a.
\end{cases}
\]
The inequality gives
\[
\int_{-\infty}^{\infty} e^{-|a-x|} f(x)\,dx \le 1.
\]
Since \(f(x)\ge 0\), restricting the integration to \([a,b]\) only makes the integral smaller. In particular,
\[
\int_{a}^{b} e^{-(x-a)} f(x)\,dx \;\le\; \int_{-\infty}^{\infty} e^{-|a-x|} f(x)\,dx \;\le\; 1.
\]
Thus 
\[
I_1 := \int_{a}^{b} e^{-(x-a)} f(x)\,dx \le 1.
\]

Next, set \(t=b\). A similar argument yields
\[
I_2 := \int_{a}^{b} e^{-(b-x)} f(x)\,dx \le 1.
\]

Now add these two inequalities:
\[
\int_{a}^{b} \bigl[e^{-(x-a)} + e^{-(b-x)}\bigr] f(x)\,dx \le 2.
\]
Meanwhile, observe that for each \(x\in [a,b]\), 
\[
2 = (2 - e^{-(x-a)} - e^{-(b-x)}) + (e^{-(x-a)} + e^{-(b-x)}).
\]
Integrate both sides over \([a,b]\) against \(f(x)\), which gives
\[
2\int_{a}^{b} f(x)\,dx \;=\; \int_{a}^{b} \bigl[2 - e^{-(x-a)} - e^{-(b-x)}\bigr]f(x)\,dx 
\;+\; \int_{a}^{b} \bigl[e^{-(x-a)} + e^{-(b-x)}\bigr]f(x)\,dx.
\]
We have already bounded the second integral on the right by \(2\). For the first integral, note that \(2 - e^{-(x-a)} - e^{-(b-x)} \le 2\) always (since the exponential terms are nonnegative). More precisely, one can show
\[
2 - e^{-(x-a)} - e^{-(b-x)} \;\le\; (b-a)
\]
for all \(x\in[a,b]\), because \(1 - e^{-u} \le u\) for \(u\ge 0\). Adding the bounds, we get
\[
2\int_{a}^{b} f(x)\,dx \;\le\; (b-a) + 2.
\]
Hence 
\[
\int_{a}^{b} f(x)\,dx \;\le\; \frac{b-a+2}{2},
\]
as required.
\end{solution}

\end{document}
